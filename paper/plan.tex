\documentclass[12pt]{article}

\def\magyarOptions{defaults=hu-min}
\usepackage[magyar]{babel}

\usepackage{t1enc}
\usepackage[utf8]{inputenc}

\usepackage{times}

\usepackage{amsmath}
\usepackage{amssymb}
\usepackage{amsthm}

\usepackage{fancyhdr}

\usepackage{hyperref}
\hypersetup{
    colorlinks,
    citecolor=black,
    filecolor=black,
    linkcolor=black,
    urlcolor=black
}

\usepackage{enumerate}
\usepackage{enumitem}
\usepackage{wrapfig}

\usepackage[labelfont=it,justification=centering]{caption}

\usepackage{graphicx}
\usepackage{epigraph}

\usepackage{dirtree}

% Margók:
\hoffset -1in
\voffset -1in
\oddsidemargin 35mm
\textwidth 150mm
\topmargin 15mm
\headheight 10mm
\headsep 5mm
\textheight 237mm

% Becsült oldaszámok követése
\usepackage{calc}
\newcounter{osszesen}
\newcommand*{\oldalak}[1]{#1 oldal \setcounter{osszesen}{\theosszesen +
#1}($\Sigma=\theosszesen$)}

\begin{document}

\clearpage
\thispagestyle{empty}
\begin{center}

\vspace*{4.8cm}

{\LARGE\bf BSc. Szakdolgozat tervezet}

\vspace*{12cm}

{\large \raggedright
\emph{Készítette:} \textbf{Ledán Szilárd} informatika szakos hallgató\\
\emph{Témavezető:} \textbf{Dr. Kiss Ákos} adjunktus\\
Szegedi Tudományegyetem, Szoftverfejlesztés Tanszék\\
}
\end{center}

\clearpage

%%%%%%%%%%%%%%%%%%%%%%%%%%%%%%%%%%%%%%%%%%%%%%%%%%%%%%%%%%%%%%%%%%%%%%
%%   Tartalomjegyzék                                                %%
%%%%%%%%%%%%%%%%%%%%%%%%%%%%%%%%%%%%%%%%%%%%%%%%%%%%%%%%%%%%%%%%%%%%%%

\tableofcontents

%%%%%%%%%%%%%%%%%%%%%%%%%%%%%%%%%%%%%%%%%%%%%%%%%%%%%%%%%%%%%%%%%%%%%%
%%   Téma                                                           %%
%%%%%%%%%%%%%%%%%%%%%%%%%%%%%%%%%%%%%%%%%%%%%%%%%%%%%%%%%%%%%%%%%%%%%%

\section{Téma}

A \emph{W3C} (World Wide Web Consortium) által \emph{Canvas 2D Context} néven
2015-ben elfogadott API ajánlás \emph{Path} részének \emph{OpenGL-ES 2.0} alapú
megvalósítása.

%%%%%%%%%%%%%%%%%%%%%%%%%%%%%%%%%%%%%%%%%%%%%%%%%%%%%%%%%%%%%%%%%%%%%%
%%   Szakdolgozat felépítése                                        %%
%%%%%%%%%%%%%%%%%%%%%%%%%%%%%%%%%%%%%%%%%%%%%%%%%%%%%%%%%%%%%%%%%%%%%%

\section{Szakdolgozat felépítése}

\subsection*{Címoldal}

\begin{description}[noitemsep]
  \item[Cím] Még pontosítani kell Ákossal.
  \item[Dátum] Javítani és ellenőrizni kell.
  \item[Témavezető] Ákos egyetemi címét is ellenőrizni kell, ha addig változna.
  \item[Oldalszám] \oldalak{1}.
\end{description}

\subsection*{Tartalomjegyzék}

\begin{itemize}[noitemsep]
  \item[\checkmark] Ez kész.
  \item[$\bullet$] Oldalszám: \oldalak{1}.
\end{itemize}

\subsection*{Feladatkiírás}

\begin{itemize}[noitemsep]
  \item[$\square$] Még meg kell írni, és Ákossal átnézetni.
  \item[$\bullet$] Oldalszám: \oldalak{1}.
\end{itemize}

\subsection*{Tartalmi összefoglaló}

\begin{itemize}[noitemsep]
  \item[$\square$] Ez is nagyjából kész. még igazítani kell a végső szöveghez.
  \item[$\bullet$] Oldalszám: \oldalak{1}.
\end{itemize}

\subsection*{Bevezetés}

\begin{itemize}[noitemsep]
  \item[\checkmark] Ez kész.
  \item[$\bullet$] Oldalszám: \oldalak{1}.
\end{itemize}

\subsection{Háttér }

Az alapvető gondolat az, hogy bemutatom a Canvas API-t mint interface-t, és
ezáltal a kényszereket is, amiknek meg kell felelnie az ezt implementáló
Gepárdnak. Azután a használni kívánt GLES2-es API-t ismertetem, hogy
milyen lehetőségek állnak a rendelkezésemre. Ezáltal érthetővé válik,
hogy mit és mivel szeretnék összekötni, és hogy mik a nehézségek.
Azután még beszélnem kell itt a szükséges és választott matematikai
háttérről is.

\subsubsection{W3C - Canvas 2D Context}

Bevezetésként leírok egy minimális történeti részt és hogy miért választottam
ezt az API-t. Azután felületesen megismertetem az olvasót az API-val, hogyan
szokták használni, mire jó, mire nem alkalmas.

\begin{description}[noitemsep]
  %\setlength{\itemsep}{0pt plus 1pt}
  \item[Kulcsszavak] Canvas, bitmap, path, zárt alakzat, fill, stroke.
  \item[Becsült oldalszám] \oldalak{3}.
\end{description}

\subsubsection{OpenGL-ES 2.0}

Röviden ismertetem az olvasóval, hogy mi is a GLES2 és miért választottam.
Majd megmutatom a számomra legfontosabb részeket és tulajdonságokat. Itt
fogom bemutatni, hogy alapvetően csak \emph{háromszögek} kitöltésére
hagyatkozhatunk. Ezért már itt előre vetítem, hogy a megoldásom egy olyan
program kell legyen, ami végül a tetszőleges alakzatot háromszögekkel lefedi.
Ezután a háromszögek gyors és a GLES2 működéséhez jól illeszkedő
módon való kirajzolásához szükséges további GLES2-es fogalmakat ismertetem.

\begin{description}[noitemsep]
  \item[Kulcsszavak] Állapotgép, shader, vertex, fragment, pipeline, batching
  queue, texturák.
  \item[Becsült oldalszám] \oldalak{4}.
\end{description}

\subsubsection{Matematikai háttér}

Az előbbi részekben már ismertetett kényszer miatt leírom, hogy az egyik
régi megközelítést fogom kissé tágabban értelmezve használni. Így a
következő egymásra épülő és használt matematikai fogalmakkal kell
megismerkedni.

\begin{description}
  \item[Seprő egyenesek] Mit értünk pontosan matematikailag seprő
  egyeneseknek, hogyan értelmezhetjük ezeket \emph{seprő sávokként}, és ez
  miért lesz hasznos számunkra.

  \begin{description}[noitemsep]
    \item[Kulcsszavak] Seprő egyenes, seprő sáv.
    \item[Becsült oldalszám] \oldalak{1}.
  \end{description}

  \item[Bézier-görbe] A másod- és harmadrendű Bézier-görbék
  bemutatása. A görbe felosztásának ismert gyors eljárása a De Casteljau
  algoritmusa, amit használunk, hogy közel szakasz-szerű görbedarabokkal
  közelítsük az eredetit.

  \begin{description}[noitemsep]
    \item[Kulcsszavak] Másodrendű-, harmadrendű Bézier-görbe, De
    Casteljau algoritmusa.
    \item[Becsült oldalszám] \oldalak{2}.
  \end{description}

  \item[Kör visszavezetése Bézier-görbékre] Itt támaszkodunk egy cikkre,
  alapvetően annak a számunkra legfontosabb tartalmát vesszük át.

  \begin{description}[noitemsep]
    \item[Kulcsszavak] Kör, körív.
    \item[Becsült oldalszám] \oldalak{2}.
  \end{description}

\end{description}

\subsection{Belső Path API}

Itt ismertetem meg az olvasóval a \emph{Gepard} rajzolót, a kitűzött
célokat, a támogatni kívánt rajzoló meghajtókat (GLES2, Vulkan).
Bemutatom a Gepard minimális felépítését, s ezen belül térek ki arra a
részre, ami az én szakdolgozatom java részét érinti, a Belső Path API és
annak GLES2-es megvalósítása.


\begin{description}[noitemsep]
  \item[Kulcsszavak] Gepard, OpenGL-ES 2.0, Vulkan, Path.
  \item[Becsült oldalszám] \oldalak{4}.
\end{description}

  \subsubsection{Path API}
  Részletbemenően ismertetem a Path API függvény-csoportjait,
  működésének elvét, a felhasznált technológiákat; illetve külön
  kitérek a \emph{stroke} hívás összetett metódusaira, amelyek révén
  visszavezettük \emph{fillre}. Ez a rész elég nagy, hogy akár egy kisebb
  alfejezete legyen a \emph{Path API} résznek.

  \begin{description}[noitemsep]
    \item[Kulcsszavak] Region memory modell, line, quadratic-curve,
    cubic-curve, arc, arc-to, rect, fill, stroke, transform, translate, rotate,
    scale, skew, save, restore, color.
    \item[Becsült oldalszám] \oldalak{7}.
  \end{description}

  \subsubsection{Segment approximator}
  Ez a lelke az algoritmusomnak. Beszélnem kell az algoritmus vázáról, majd
  az egyes részeket megvilágítva: mi történik a listához adott
  szakaszokkal (segments), hogyan kezelem a metszéspontokat, milyen nem
  ábrázolható folytonossági problémák merülhetnek fel metszések
  esetén, ezeket hogyan próbáltam meg orvosolni.

  \begin{description}[noitemsep]
    \item[Kulcsszavak] Szakasz - segment, metszéspontok.
    \item[Becsült oldalszám] \oldalak{6}.
  \end{description}

  \subsubsection{Trapezoid tessellator}
  Ebben e részben a halomra hányt szakaszokból a kitöltési metódus
  alapján trapézokat hozok létre. Ezt az algoritmust is részletesen be kell
  mutatnom. A trapéz építése, majd horizontális, végül vertikális
  összevonásukról is kell beszélnem.

  \begin{description}[noitemsep]
    \item[Kulcsszavak] Trapéz, horizontális összevonás, vertikális összevonás.
    \item[Becsült oldalszám] \oldalak{4}.
  \end{description}

  \subsubsection{Rajzolás GLES2-vel}
  Az rajzolás utolsó fázisa. Itt mutatom be a trapézok
  háromszög-párként való kirajzolását. Hogyan gyűjtjük egybe a
  háromszög-párokat, és hogyan rajzoljuk ki azokat együtt. Hogyan lesz
  ebből egy alpha textúra, és azt hogyan használjuk fel színes
  kirajzolásra. Itt világosítom meg, hogy miért van szükség alpha textúrára?

  \begin{description}[noitemsep]
    \item[Kulcsszavak] Batching, alpha textúra, textúrázás, színezés,
    gradiens, kivágás - clipping.
    \item[Becsült oldalszám] \oldalak{4}.
  \end{description}

\subsection{Használat és eredmények}
  \subsubsection{Használat}
  Ez a fejezet fog szólni arról, hogy hogyan kell a \emph{libgepard.so}-t
  előállítani, illetve hogyan kell használni. Bemutatok néhány egyszerű
  példát.

  \begin{description}[noitemsep]
    \item[Kulcsszavak] Gepard kontext, libgepard.
    \item[Becsült oldalszám] \oldalak{4}.
  \end{description}

  \subsubsection{Eredmények}
  Itt a lefedettséget, illetve néhány ismert rajzolóval való minőségi
  összehasonlítást fogom bemutatni.

  \begin{description}[noitemsep]
    \item[Kulcsszavak] Blink, Gecko.
    \item[Becsült oldalszám] \oldalak{4}.
  \end{description}

\subsection{Összefoglalás }
Végül összefoglalom az elért eredményeket. Illetve összegzem a még
lehetséges javításokat, gyorsításokat. Valamint a jövőbeli terveket vázolom fel.
\begin{description}[noitemsep]
  \item[Kulcsszavak] Paralel trapezoid bontás, batching queue.
  \item[Becsült oldalszám] \oldalak{2}.
\end{description}

\subsection*{Irodalomjegyzék }
\begin{itemize}[noitemsep]
  \item[\checkmark] Magától generálódik.
  \item[$\bullet$] Oldalszám: \oldalak{1}.
\end{itemize}

\subsection*{Nyilatkozat }
\begin{itemize}[noitemsep]
  \item[\checkmark] Ez kész.
  \item[$\bullet$] Oldalszám: \oldalak{1}.
\end{itemize}

\subsection*{Köszönetnyilvánítás }
\begin{itemize}[noitemsep]
  \item[$\square$] Még meg kell írni.
  \item[$\bullet$] Oldalszám: \oldalak{1}.
\end{itemize}

%%%%%%%%%%%%%%%%%%%%%%%%%%%%%%%%%%%%%%%%%%%%%%%%%%%%%%%%%%%%%%%%%%%%%%
%%   Mappaszerkezet                                                 %%
%%%%%%%%%%%%%%%%%%%%%%%%%%%%%%%%%%%%%%%%%%%%%%%%%%%%%%%%%%%%%%%%%%%%%%

\section{Mappaszerkezet}
\dirtree{%
.1 gepard.
.2 code.
.3 gepard: behúzva külsős projektként.
.2 paper.
.3 examples: a szakdolgozatban használt példák.
.3 src: a szakdolgozat forrása.
.4 bib: bibliográfia.
.4 img: a felhasznált képek.
.4 svg: szerkesztett SVG fájlok.
.4 \textbf{ledanszilard.tex}.
.4 \textbf{kotestabla.tex}.
.2 scripts: hasznos szkriptek, konvertáláshoz, ellenőrzéshez, stb.
}

\end{document}
